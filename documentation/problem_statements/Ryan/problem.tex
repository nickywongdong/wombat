\documentclass[onecolumn, draftclsnofoot,10pt, compsoc]{IEEEtran}
\usepackage{graphicx}
\usepackage{url}
\usepackage{setspace}

\usepackage{geometry}
\geometry{textheight=9.5in, textwidth=7in}

% 1. Fill in these details
%\def \CapstoneTeamName{		The Jetsons}
\def \CapstoneTeamNumber{		15}
\def \GroupMemberOne{			Ryan Crane}
\def \GroupMemberTwo{			Victor Li}
\def \GroupMemberThree{			Nicholas Wong}
\def \CapstoneProjectName{		Nvidia Jetson TX2 based infotainment and black box}
\def \CapstoneSponsorCompany{	OSU}
\def \CapstoneSponsorPerson{		D. Kevin McGrath}

% 2. Uncomment the appropriate line below so that the document type works
\def \DocType{		Problem Statement
				%Requirements Document
				%Technology Review
				%Design Document
				%Progress Report
				}

\newcommand{\NameSigPair}[1]{\par
\makebox[2.75in][r]{#1} \hfil 	\makebox[3.25in]{\makebox[2.25in]{\hrulefill} \hfill		\makebox[.75in]{\hrulefill}}
\par\vspace{-12pt} \textit{\tiny\noindent
\makebox[2.75in]{} \hfil		\makebox[3.25in]{\makebox[2.25in][r]{Signature} \hfill	\makebox[.75in][r]{Date}}}}
% 3. If the document is not to be signed, uncomment the RENEWcommand below
\renewcommand{\NameSigPair}[1]{#1}

%%%%%%%%%%%%%%%%%%%%%%%%%%%%%%%%%%%%%%%

\begin{document}

\begin{titlepage}
    \pagenumbering{gobble}
    \begin{singlespace}
    	\includegraphics[height=4cm]{coe_v_spot1}
        \hfill
        % 4. If you have a logo, use this includegraphics command to put it on the coversheet.
        %\includegraphics[height=4cm]{CompanyLogo}
        \par\vspace{.2in}
        \centering
        \scshape{
            \huge CS Capstone\DocType \par
            {\large\today}\par
            \vspace{.5in}
            \textbf{\Huge\CapstoneProjectName}\par
            \vfill
            {\large Prepared for}\par
            \Huge \CapstoneSponsorCompany\par
            \vspace{5pt}
            {\Large\NameSigPair{\CapstoneSponsorPerson}\par}
            {\large Prepared by }\par
            Group\CapstoneTeamNumber\par
            % 5. comment out the line below this one if you do not wish to name your team
	    %\CapstoneTeamName\par
            \vspace{5pt}
            {\Large
                \NameSigPair{\GroupMemberOne}\par
                \NameSigPair{\GroupMemberTwo}\par
                \NameSigPair{\GroupMemberThree}\par
            }
            \vspace{20pt}
        }
        \begin{abstract}
        % 6. Fill in your abstract
        	The goal of this project is to develop an infotainment system for a car with a black box that logs
					GPS and orientation data. This device will be based on Nvidia's Jetson TX2 platform. It should
					have a touch screen interface with a GUI and access to FM radio, music from an auxiliary port,
					and navigation information. The "black box" portion of the project requires AHRS sensors
					and coordinating multiple GPS sensors to keep an accurate log of the car's recent history.
        \end{abstract}
    \end{singlespace}
\end{titlepage}
\newpage
\pagenumbering{arabic}
%\tableofcontents
% 7. uncomment this (if applicable). Consider adding a page break.
%\listoffigures
%\listoftables
\clearpage

% 8. now you write!
\section{Problems}
Our task is to assemble an infotainment console and black box system with multiple sensors for installation in a car,
 and to build and source necessary software to run its systems. The infotainment console should have a touchscreen
  interface with a functionality to control normal vehicle media, and view navigation information.
	The black box should communicate with multiple GPS sensors and  an attitude and heading reference system,
	 and generate a readable log of its recent history. All of this will run off of an Nvidia Jetson TX2.
A user interface will have to be designed to make the systems features accessible. The system
should have access to FM radio, and the user should be able to see what they are doing.
The system should also interpret RDS to display information encoded in broadcasts.
It should also be able to play music from external devices via an auxiliary port. It has been
determined that certain functions such as climate control and volume should be controllable by
physical knobs, rather than exclusively through the touch screen.
The black box component needs to coordinate several sensors around the car to
produce a log of GPS and orientation information from the cars recent usage.
The log should be relatively easy to understand. The control console should have access
to the log and offer users the ability to clear the log, though it should be made difficult to do by
accident to avoid unwanted data loss. Data presentation is important and it would be beneficial if sensor
data could be analyzed to give drivers feedback on fuel economy, potentially with some gamification.
The screen should also display navigational information. The project will likely use OpenStreetMap for
mapping information. The system should be able to operate offline as it is important to remain usable
in remote and off road settings. It would be desirable, although not required, to implement topographical
mapping data as well.
Most elements of the infotainment portion of this project are features that
either are present or are not, and the metric for success is essentially a
check box. Targets should be established for the time window covered by the data log.
The navigation data needs to be reasonably accurate. The GPS and attitude logs should
 also be measurably accurate. 

\end{document}
