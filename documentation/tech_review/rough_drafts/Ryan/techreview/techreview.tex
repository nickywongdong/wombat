\documentclass[onecolumn, draftclsnofoot,10pt, compsoc]{IEEEtran}
\usepackage{graphicx}
\usepackage{url}
\usepackage{setspace}

\usepackage{geometry}
\geometry{textheight=9.5in, textwidth=7in}

% 1. Fill in these details
\def \CapstoneTeamName{		Team Wombat}
\def \CapstoneTeamNumber{		15}
\def \GroupMemberOne{			Ryan Crane}
\def \GroupMemberTwo{			Victor Li}
\def \GroupMemberThree{			Nicholas Wong}
\def \CapstoneProjectName{		Axolotl}
\def \CapstoneSponsorCompany{	OSU}
\def \CapstoneSponsorPerson{		D. Kevin McGrath}

% 2. Uncomment the appropriate line below so that the document type works
\def \DocType{		%Problem Statement
				%Requirements Document
				Technology Review
				%Design Document
				%Progress Report
				}

\newcommand{\NameSigPair}[1]{\par
\makebox[2.75in][r]{#1} \hfil 	\makebox[3.25in]{\makebox[2.25in]{\hrulefill} \hfill		\makebox[.75in]{\hrulefill}}
\par\vspace{-12pt} \textit{\tiny\noindent
\makebox[2.75in]{} \hfil		\makebox[3.25in]{\makebox[2.25in][r]{Signature} \hfill	\makebox[.75in][r]{Date}}}}
% 3. If the document is not to be signed, uncomment the RENEWcommand below
%\renewcommand{\NameSigPair}[1]{#1}

%%%%%%%%%%%%%%%%%%%%%%%%%%%%%%%%%%%%%%%
\begin{document}
\begin{titlepage}
    \pagenumbering{gobble}
    \begin{singlespace}
    	\includegraphics[height=4cm]{coe_v_spot1}
        \hfill
        % 4. If you have a logo, use this includegraphics command to put it on the coversheet.
        %\includegraphics[height=4cm]{CompanyLogo}
        \par\vspace{.2in}
        \centering
        \scshape{
            \huge CS Capstone \DocType \par
            {\large\today}\par
            \vspace{.5in}
            \textbf{\Huge\CapstoneProjectName}\par
            \vfill
            {\large Prepared for}\par
            \Huge \CapstoneSponsorCompany\par
            \vspace{5pt}
            {\Large\NameSigPair{\CapstoneSponsorPerson}\par}
            {\large Prepared by }\par
            Group\CapstoneTeamNumber\par
            % 5. comment out the line below this one if you do not wish to name your team
            \CapstoneTeamName\par
            \vspace{5pt}
            {\Large
                \NameSigPair{\GroupMemberOne}\par
%                \NameSigPair{\GroupMemberTwo}\par
%                \NameSigPair{\GroupMemberThree}\par
            }
            \vspace{20pt}
        }
				\begin{abstract}
			         % 6. Fill in your abstract
			         	This document investigates GPS Modules, backup cameras and Wireless
								video streaming solutions for use in the Axolotl project. A GPS module
								is necessary to deliver navigation and route guidance. Backup camera
								integration is a required feature and wireless streaming is
								desirable to reduce the amount of wiring in the project. 
			         \end{abstract}
    \end{singlespace}
\end{titlepage}
\newpage
\pagenumbering{arabic}
\tableofcontents
% 7. uncomment this (if applicable). Consider adding a page break.
%\listoffigures
%\listoftables
\clearpage

% 8. now you write!
\section{Overview}
The Axolotl is a car infotainment and navigation system paired with a black box.
It will provide access to mapping, route guidance, music, and cameras mounted
to the dashboard or the rear of the car. Users will interact with the system
through a touch screen interface which will also be used as displays for the
cameras. The black box system will keep a log of GPS and AHRS data as well as
video from the cameras and some data captured from the car's internal computer.
This technology review will examine GPS modules, backup cameras, and wireless
video streaming solutions that can be used to complete this project.

\section{GPS Modules}
It will be necessary to select a GPS module to link to the Axolotl in order
to provide navigation services. To choose a module it will be important to
consider the update rate of a module. Update rate measure how frequently a
module calculates and posts its position. Modules have update rates ranging
from 1 Hz to 20 Hz and above. The number of channels a module uses to find a
satellite to talk to is also important. Checking more channels allows a module
to determine its position faster, especially when it first starts up. Twelve to
fourteen channels is a common baseline for GPS Modules.

One possible GPS module is the Copernicus 2 from Trimble. This module has a on
Hz update rate and uses twelve channels to communicate with satellites.
It normally takes about 38 seconds to lock its position on startup. The
Copernicus is accurate within 5 to 10 meters and is capable of enduring harsh
temperature, humidity and vibration.

Another option is the Venus638 from SkyTraq. This module has an update rate up
to 20Hz and utilizes 65 channels to communicate with GPS satellites. It has a
slightly faster startup time at 29 seconds and is accurate within approximately
2.5 meters. Additionally the Venus unit has jamming detection and mitigation
systems to help drown out noise and increase reliability.

The third GPS module is the NEO M8U from u-blox. The M8U also has a 20 Hz update
rate and has a shorter startup time of 26 seconds with a 72 channel receiver.
This unit also has an integrated accelerometer, gyroscope and odometer. Most
attractively it has built in untethered dead reckoning functionality. Dead
reckoning is when a system extrapolates its position based on other data after
losing its GPS signal. It is very useful when traveling under thick cover or in
remote areas.

The Copernicus is a very basic module, it is functional but it is generally
sluggish compared to the other two. It has a slower startup speed and a lower
update rate, making it less than ideal. The Venus is much sharper than the
Copernicus and its anti jamming functions could be helpful in noisy areas.
The M8U Module has the best performance overall and its integrated dead reckoning
support is a major bonus because native support for it could greatly reduce
development time and complexity.

We chose the M8U Module because its dead reckoning support will be very valuable
to this project. Dead reckoning will allow the Axolotl to more smoothly move
through areas without a GPS signal, such as tunnels. Because of the built in
support for dead reckoning in the M8U module It will ease the complexity and
time required to implement the feature.

\section{Backup Cameras}
This project also requires use of a backup camera mounted to the car. Backup
cameras are generally a great feature for cars and the government will require
new cars to have them beginning in 2018. The backup camera will be displayed on
the screen when the car is in reverse. Video from the backup camera might also
be included in the log.

Important things to consider in backup cameras include viewing angle and low
light capability. Viewing angle describes how far to either side the camera will
see. Low light functionality is important when parking at night. Some cameras
have night vision or lights to illuminate the ground nearby. There are also
several different mounting styles for backup cameras to be considered. Some
cameras mount to the license plate, others have brackets, or mount flush with
the car after a hole is drilled for them.

One backup camera option is BOYO's bracket mount camera. It features a simple
bracket mount and has night vision for low light driving. This camera also has
a built in guide line to help illustrate distances in the video to assist
parking. The cameras output resolution is 640 x 480.

Another camera option is a license plate mounted camera from Metra. This
camera Has a very wide viewing angle of 170 degrees, and also has night vision
like the previous camera. Metra also provides parking assist guidelines on the
display.

A third option is SVTCAM's backup camera. This camera boasts 720p resolution
and 175 degree viewing angle. Like the others this camera offers night vision
and parking assist guidelines.

It turns out that backup cameras have a fairly universal feature set, the primary
difference being the mounting style. For testing purposes it seems ideal to use
a mount that is easy to install and remove without being destructive. A flush
mount camera for example would not be ideal as it requires a hole to be drilled
or sawed for normal installation.

We recommend the Metra camera because the license plate mount seems like
it would be the most versatile style to test with. Normal installation of
bracket mount cameras requires them to be screwed in place. We could use tape or
another adhesive instead, but it might not be as secure.

\section{Wireless Video Streaming}
To avoid running long wires through the car we need to set up a wireless
connection to stream video from the backup camera to the main Axolotl unit.
Wireless streaming makes installation and maintenance much easier, since it
would be quite difficult to run a video wire the length of the car in a way
that doesn't look terrible.

When looking at wireless streaming solutions it is important to consider price
and the complexity of development and implementation. Due to the scope of the
project and the number features we need to implement it would be undesirable
for video streaming to require a large amount of development time.

One way to deliver wireless video is with a dedicated video transmitter and
receiver. This is potentially the least costly solution in both money and
development time. Transmitter receiver pairs can be found for as little as ten
to twenty dollars, and in theory it should just work when you plug them in.
When implemented in a car the transmitter may require some additional
weatherproofing since will likely be somewhat exposed to the elements.

An alternative option is to connect the camera to a small computer like a
Raspberry Pi, which could control the camera and send video over wifi to our
Jetson. This is a little more expensive, Raspberry Pi computers as an example
cost 25 to 35 dollars. This would also require some programming and the computer
would definitely need to be protected when implemented, making it less time
efficient as well. This option would allow more control over the camera then
the previous solution.

A third option for wireless video would be to simply buy a wireless camera. This
is the most expensive solution as wireless cameras start at around one hundred
dollars. Additionally in most cases these products come with a monitor and offer
little information as to how they talk. Because of this it is difficult to
predict what might be necessary to integrate a camera into our system, or if it
can be done.

 These three solutions contrast quite starkly. One is cheap and easy, one is
 expensive and potentially very difficult and one is middle of the road on both
 counts. The Raspberry Pi, unlike the other two solutions, could allow us more
 fine control of the camera. Normally when a backup camera is installed it is
 wired to the backup lights so it only turns on when the car is in reverse. With
 a Raspberry Pi in place to control the camera we could add implement additional
 features such as allowing users to view the camera at will.

 We believe a secondary controller for the computer is the best choice because
 of the flexibility it adds to the system. A secondary controller could allow us
to give the end user greater ability to manipulate the camera. It would also make
it easier to incorporate footage from the backup camera into the black box log.

\nocite{*}
\bibliographystyle{plain}
\bibliography{writing}

\end{document}
