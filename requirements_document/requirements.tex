\documentclass[onecolumn, draftclsnofoot,10pt, compsoc]{IEEEtran}
\usepackage{graphicx}
\usepackage{url}
\usepackage{setspace}

\usepackage{geometry}
\geometry{textheight=9.5in, textwidth=7in}

\usepackage{graphicx}
\graphicspath{{images/}}

% 1. Fill in these details
\def \CapstoneTeamName{		Team Wombat}
\def \CapstoneTeamNumber{		15}
\def \GroupMemberOne{			Victor Li}
\def \GroupMemberTwo{			Ryan Crane}
\def \GroupMemberThree{			Nicholas Wong}
\def \CapstoneProjectName{		Axolotl}
\def \CapstoneSponsorCompany{	}
\def \CapstoneSponsorPerson{		Kevin McGrath}

% 2. Uncomment the appropriate line below so that the document type works
\def \DocType{		%Problem Statement
				Requirements Document
				%Technology Review
				%Design Document
				%Progress Report
				}
			
\newcommand{\NameSigPair}[1]{\par
\makebox[2.75in][r]{#1} \hfil 	\makebox[3.25in]{\makebox[2.25in]{\hrulefill} \hfill		\makebox[.75in]{\hrulefill}}
\par\vspace{-12pt} \textit{\tiny\noindent
\makebox[2.75in]{} \hfil		\makebox[3.25in]{\makebox[2.25in][r]{Signature} \hfill	\makebox[.75in][r]{Date}}}}
% 3. If the document is not to be signed, uncomment the RENEWcommand below
%\renewcommand{\NameSigPair}[1]{#1}

%%%%%%%%%%%%%%%%%%%%%%%%%%%%%%%%%%%%%%%
\begin{document}
\begin{titlepage}
    \pagenumbering{gobble}
    \begin{singlespace}
    	\includegraphics[height=4cm]{coe_v_spot1}
        \hfill 
        % 4. If you have a logo, use this includegraphics command to put it on the coversheet.
        %\includegraphics[height=4cm]{CompanyLogo}   
        \par\vspace{.2in}
        \centering
        \scshape{
            \huge CS Capstone \DocType \par
            {\large\today}\par
            \vspace{.5in}
            \textbf{\Huge\CapstoneProjectName}\par
            \vfill
            {\large Prepared for}\par
            \Huge \CapstoneSponsorCompany\par
            \vspace{5pt}
            {\Large\NameSigPair{\CapstoneSponsorPerson}\par}
            {\large Prepared by }\par
            Group\CapstoneTeamNumber\par
            % 5. comment out the line below this one if you do not wish to name your team
            \CapstoneTeamName\par 
            \vspace{5pt}
            {\Large
                \NameSigPair{\GroupMemberOne}\par
                \NameSigPair{\GroupMemberTwo}\par
                \NameSigPair{\GroupMemberThree}\par
            }
            \vspace{20pt}
        }
        %\begin{abstract}
        % 6. Fill in your abstract    
   % Our project requires the development of software to run a car infotainment system alongside a black box data log. Our software will run on an Nvidia Jetson TX2 and should be able to execute basic media and navigation tasks while also logging data from a variety of sensors. The goal is to allow people to modernize older cars with an array of fresh sensors and an interactive software interface. To accomplish this goal, we will pull vehicle data from a car?s OBDII (On Board Diagnostics II) port as well as our own sensors wired into the TX2, and feed that data to software that aggregate that data into a log. 
      %  \end{abstract}     
    \end{singlespace}
\end{titlepage}
\newpage
\pagenumbering{arabic}
\tableofcontents
% 7. uncomment this (if applicable). Consider adding a page break.
%\listoffigures
%\listoftables
\clearpage

% 8. now you write!
\section{Introduction}
\subsection{Purpose}
The purpose of this software requirements specification (SRS) document is to outline and detail the capabilities of the NVIDIA Jetson TX2 infotainment and black box our group will develop, known henceforth as the Axolotl Infotainment System and Axolotl OS. Doing so will enable us to describe the requirements of the Axolotl Infotainment System and Axolotl OS such that we and our client will have a detailed understanding of the form factor and capabilities of the deliverable system we will develop. The intended audience for this SRS includes our client, the CS Capstone Instructors, and our group.\par

\subsection{Scope}
Our project entails the development of an infotainment system and black box that can be divided into two products: the Axolotl and the Axolotl Software. The Axolotl will connect vehicle sensors, controllers, receivers, and a touchscreen to a NVIDIA Jetson TX2 computer in a package that can be installed in a vehicle. The Axolotl Software runs on the Axolotl and provides users with media playback, navigation, and vehicle data logging capabilities.\par


\section{Definitions}


\section{Performance Metrics}
We will evaluate our performance on this project by examining whether or not we have fulfilled a variety of requirements that will be divided into "necessities" and "superlatives".\par
"Necessities" outline the minimum functionality that is expected of our infotainment system and black box requested by our client; not fulfilling these will amount to failure of the project; completion of all necessities amounts to the completion of the project. "Necessities" will be considered our base requirements in software development. An example of a ?necessity? requirement would be: implementation of FM (frequency modulation) radio with functional tuning, audio playback, and display of RDS (radio data system) data onto the touchscreen for the current radio channel.\par
"Superlatives" will outline additional features that are not explicitly required to fulfill the requested functionality of the project, but are desirable (or nice-to-have) and will make our system more capable and versatile.\par

\subsection{Fulfillment}
Successful implementation of necessities and superlatives can be evaluated based on the presence or absence of the feature with regard to the requirements specified. This is because there is little to no middle ground with much of the functionality; for example, data logging cannot "somewhat work": it either functions or it doesn't. This makes it easier to write requirements, test the implementation of said requirements, and then subsequently gauge the completeness of the implementation.

\end{document}