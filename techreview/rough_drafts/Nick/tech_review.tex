\documentclass[onecolumn, draftclsnofoot,10pt, compsoc]{IEEEtran}
\usepackage{graphicx}
\usepackage{url}
\usepackage{setspace}

\usepackage{geometry}
\geometry{textheight=9.5in, textwidth=7in}

% 1. Fill in these details
\def \CapstoneTeamName{		Wombat}
\def \CapstoneTeamNumber{		15}
\def \GroupMemberOne{			Victor Li}
\def \GroupMemberTwo{			Ryan Crane}
\def \GroupMemberThree{			Nick Wong}
\def \CapstoneProjectName{		Axolotl }
\def \CapstoneSponsorCompany{	}
\def \CapstoneSponsorPerson{		Kevin McGrath}

% 2. Uncomment the appropriate line below so that the document type works
\def \DocType{	%Problem Statement
				%Requirements Document
				Technology Review
				%Design Document
				%Progress Report
				}
			
\newcommand{\NameSigPair}[1]{\par
\makebox[2.75in][r]{#1} \hfil 	\makebox[3.25in]{\makebox[2.25in]{\hrulefill} \hfill		\makebox[.75in]{\hrulefill}}
\par\vspace{-12pt} \textit{\tiny\noindent
\makebox[2.75in]{} \hfil		\makebox[3.25in]{\makebox[2.25in][r]{Signature} \hfill	\makebox[.75in][r]{Date}}}}
% 3. If the document is not to be signed, uncomment the RENEWcommand below
%\renewcommand{\NameSigPair}[1]{#1}

%%%%%%%%%%%%%%%%%%%%%%%%%%%%%%%%%%%%%%%
\begin{document}
\begin{titlepage}
    \pagenumbering{gobble}
    \begin{singlespace}
    	\includegraphics[height=4cm]{coe_v_spot1}
        \hfill 
        % 4. If you have a logo, use this includegraphics command to put it on the coversheet.
        %\includegraphics[height=4cm]{coe_v_spot1}   
        \par\vspace{.2in}
        \centering
        \scshape{
            \huge CS Capstone \DocType \par
            {\large\today}\par
            \vspace{.5in}
            \textbf{\Huge\CapstoneProjectName}\par
            \vfill
            {\large Prepared for}\par
            \Huge \CapstoneSponsorCompany\par
            \vspace{5pt}
            {\Large\NameSigPair{\CapstoneSponsorPerson}\par}
            {\large Prepared by }\par
            Group\CapstoneTeamNumber\par
            % 5. comment out the line below this one if you do not wish to name your team
            \CapstoneTeamName\par 
            \vspace{5pt}
            {\Large
                \NameSigPair{\GroupMemberOne}\par
                \NameSigPair{\GroupMemberTwo}\par
                \NameSigPair{\GroupMemberThree}\par
            }
            \vspace{20pt}
        }
    \end{singlespace}
\end{titlepage}
\newpage
\pagenumbering{arabic}
\tableofcontents
% 7. uncomment this (if applicable). Consider adding a page break.
%\listoffigures
%\listoftables
\clearpage

% 8. now you write!

\section{Media Player Software}
\subsection{Overview}
In the product Axolotl, there is a need for software that allows for the playing of media files. This software should be portable and compatible for the Axolotl OS.

\subsection{Criteria}
Things that should be considered when choosing an open source media player:
\begin{itemize}
    \item should support at least .mp3 files
    \item should have the ability to automatically play the next song within a folder
    \item can be implemented into the Axolotl interface directly
    \item should have an interface that matches the rest of Axolotl OS
\end{itemize}

\subsection{Kodi Software}
Kodi is an open source media player application that has support for most devices and operating systems. Kodi also allows for their users to develop add-ons that may allow for different web services to be used. There is a large library of skins that may be applied to the software itself to match the interface of the Axolotl OS. The supported audio formats include almost all common types including mp3, flac, wav, and wma. Kodi also has support for the playing of media files from local and network storage mediums. The Kodi software has already been used as the media software in a car infotainment system known as CarPC. 

Pros:
\begin{itemize}
    \item built in support to apply different skins
    \item support for many audio formats
    \item user defined add ons for difference services
\end{itemize}

Cons:
\begin{itemize}
    \item none yet
\end{itemize}

\subsection{Clementine}
Clementine is a cross platform music player. It features a queue manager which allows temporary playlists. Clementine also has built in support for songs uploaded to certain web services including: Box, Dropbox, Google Drive, and OneDrive. It features a music visualizer with support for unix systems. The media files that are supported depend on the codecs that are on the system at the time. 

Pros:
\begin{itemize}
    \item small and clean: strictly music player
    \item queue manager
    \item visualizer
    \item built in support for media player through web services
    \item easy to understand interface
\end{itemize}

Cons:
\begin{itemize}
    \item native supported music formats unspecified
\end{itemize}

\subsection{VLC Media Player}
VLC media player is the most well known media player for all platforms. It has a
large library of documentation and supported features. It has support for almost
all media formats. The source code is readily available on Git. VLC has a media
framework which is ready to be embedded into any application to add multimedia
capabilities.  

Pros:
\begin{itemize}
    \item large community with knowledge for development
    \item source code readily available
    \item most support for media formats
\end{itemize}

Cons:
\begin{itemize}
    \item will have to find matching interface
\end{itemize}

\subsection{Discussion}
The media player that we'll most likely use is the Kodi Software.
Kodi and VLC seem to have support for the most common music file formats natively.
In contrast, Clementine seems to require further codec addition to allow for as
much support as VLC or Kodi. 

Clementine has a queue manager which allows for the user to select a song to play
next without disturbing the current media. In contrast, neither Kodi or VLC seem to
have this kind of function, but it can be written when implementing the media
software within the Axolotl OS. 

Although the Clementine already has a nice interface that may match the Axolotl's
interface, Kodi allows for many different skins in case Clementine's UI doesn't
match. In contrast, VLC doesn't seem to have much support for interfaces however, 
in this case we may be able to take more time to create its own interface.

\subsection{Conclusion}
We choose the Kodi software because it has support for all the features that we will most likely need.

\section{Main Storage Medium}
\subsection{Overview}
In building the Axolotl, there is a need to have enough storage for any data that will be stored. Depending on the data stored from sensors, the largest constraint for this section would be disk space. When building the Axolotl unit itself, there may be a need to consider the physical size of the storage mediums when choosing one. The retrieval and writing of data will be run in the background especially for logging, so speeds of the storage medium is a constraint, but the lowest priority.

\subsection{Criteria}
Things that should be considered when choosing a Storage Medium:

\begin{itemize}
    \item large disk space
    \item physical constraint
    \item sufficient read/write speeds
\end{itemize}

\subsection{Hard Drives}
The most common method of storage which has superior disk space with respect to price. The speed of an HDD depends on the rate at which the drive spins. Speeds vary from 5400, to 15,000 RPM. The physical size of a hard drive can either be 2.5 inches, or 3.5 inches. 


Pros:
\begin{itemize}
    \item large amounts of disk space with respect to price
    \item many different form factors
\end{itemize}

Cons:
\begin{itemize}
    \item lower read/write speeds
\end{itemize}

\subsection{Solid State Drives}
The much faster type of storage medium with lower disk space with respect to price. The speed of an SSD far surpasses HDD's. SSD's come in many physical form factors: 5.25 inch, 3.5 inch, 2.5 inch, and 1.8 inch.

Pros:
\begin{itemize}
    \item fast read/write speeds
    \item many different form factors
\end{itemize}

Cons:
\begin{itemize}
    \item pricey
\end{itemize}

\subsection{M.2 SSD Drives}
A new type of SSD which is internally mounted. It has similar read/write speeds with the SSD. 

Pros:
\begin{itemize}
    \item allows for direct attachment to main module
\end{itemize}

Cons:
\begin{itemize}
    \item main module must support the connection type
\end{itemize}

\subsection{Discussion}
Each option has no constraint with data space, but if price becomes a factor, then HDD is clearly the best.
In terms of physical constraint, both HDD and SSD come in similar form factors.
In terms of speed, both types of the SSD are much faster in comparison to the HDD. 
However, the benefit of choosing the M.2 SSD drive is being able to mount the drive to the head unit itself in contrast to using SATA. This is only possible if the head module we choose has support for this method of connection.
\subsection{Conclusion}
The most likely option would be the HDD. Although price may not be a constraint, the product does not strictly require read/write speeds to be much faster than what an HDD offers. 

\section{Head Unit Module}
The Axolotl head unit must be built on a module which has compatibility for all the requirements mentioned.

\subsection{Overview}
What should be considered when choosing the main module:
\subsection{Criteria}
\begin{itemize}
    \item compatible with all requirements specified in requirements document
\end{itemize}

\subsection{Nvidia Jetson TX2}
The Jetson TX2 features:
\begin{itemize}
    \item NVIDIA Pascal GPU
    \item 2 Denver CPU's with the A57 microarchitecture
    \item 32 GB Flash Storage
    \item Bluetooth compatible
    \item I/O with HDMI, PCI-E x4, M.2 Key E
\end{itemize}
This is the newer version of the Jetson TX1, with faster speeds. 

\subsection{Nvidia Jetson TX1}
The Jetson TX1 geatures:
\begin{itemize}
    \item NVidia Maxwell GPU
    \item ARM Cortex A57
    \item 16 GB Flash Storage
    \item Bluetooth compatible
    \item I/O with HDMI, PCI-E x 4, M.2 Key E
\end{itemize}
This module has similar features to the TX2.

\subsection{Rasberry Pi}
The newest Rasberry Pi 3 features:
\begin{itemize}
    \item 4 x ARM Cortex A53
    \item Broadcom VideoCore IV GPU
    \item Bluetooth compatible
    \item I/O with HDMI, microSD, 3.5mm analogue audio-video jack
\end{itemize}
\subsection{Discussion}
There has already been an infotainment system that has been developed on the Rasberry Pi similar to the Axolotl. In comparison, the Jetson TX2 and TX1 will have no documentation in creating a infotainment system. 
The Rasberry Pi only has support for microSD card storage without its adapter board in contast to full support on the TX1 and TX2. 
The TX2 is vastly superior to the TX1 and the Rasberry Pi in terms of computing power. 

\subsection{Conclusion}
the Nvidia Jetson TX2 will most likely be chosen mainly because it is already available. However, the RasberryPi should be sufficient for developing something like an infotainment system.

\subsection{References}
\begin{itemize}
    \item http://engineering-diy.blogspot.ro/search/label/kodi

    \item https://www.clementine-player.org/
    \item https://kodi.tv/
    \item https://www.videolan.org/vlc/index.html

    \item https://www.seagate.com/tech-insights/choosing-high-performance-storage-is-not-about-rpm-anymore-master-ti/
    \item https://www.pcworld.com/article/2977024/storage/m2-ssd-roundup-tiny-drives-deliver-huge-performance.html

    \item https://developer.nvidia.com/embedded/buy/jetson-tx2-devkit
    \item https://developer.nvidia.com/embedded/buy/jetson-tx1-devkit
    \item https://www.raspberrypi.org/magpi/raspberry-pi-3-specs-benchmarks/
\end{itemize}

\end{document}