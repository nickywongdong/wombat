\documentclass[onecolumn, draftclsnofoot,10pt, compsoc]{IEEEtran}
\usepackage{graphicx}
\usepackage{url}
\usepackage{setspace}

\usepackage{geometry}
\geometry{textheight=9.5in, textwidth=7in}

% 1. Fill in these details
\def \CapstoneTeamName{		Team Wombat}
\def \CapstoneTeamNumber{		15}
\def \GroupMemberOne{			Victor Li}
\def \GroupMemberTwo{			Ryan Crane}
\def \GroupMemberThree{			Nicholas Wong}
\def \CapstoneProjectName{		NVIDIA Jetson TX2 Infotainment and Black Box}
\def \CapstoneSponsorCompany{	}
\def \CapstoneSponsorPerson{		Kevin McGrath}

% 2. Uncomment the appropriate line below so that the document type works
\def \DocType{		Problem Statement
				%Requirements Document
				%Technology Review
				%Design Document
				%Progress Report
				}
			
\newcommand{\NameSigPair}[1]{\par
\makebox[2.75in][r]{#1} \hfil 	\makebox[3.25in]{\makebox[2.25in]{\hrulefill} \hfill		\makebox[.75in]{\hrulefill}}
\par\vspace{-12pt} \textit{\tiny\noindent
\makebox[2.75in]{} \hfil		\makebox[3.25in]{\makebox[2.25in][r]{Signature} \hfill	\makebox[.75in][r]{Date}}}}
% 3. If the document is not to be signed, uncomment the RENEWcommand below
%\renewcommand{\NameSigPair}[1]{#1}

%%%%%%%%%%%%%%%%%%%%%%%%%%%%%%%%%%%%%%%
\begin{document}
\begin{titlepage}
    \pagenumbering{gobble}
    \begin{singlespace}
    	\includegraphics[height=4cm]{coe_v_spot1}
        \hfill 
        % 4. If you have a logo, use this includegraphics command to put it on the coversheet.
        %\includegraphics[height=4cm]{CompanyLogo}   
        \par\vspace{.2in}
        \centering
        \scshape{
            \huge CS Capstone \DocType \par
            {\large\today}\par
            \vspace{.5in}
            \textbf{\Huge\CapstoneProjectName}\par
            \vfill
            {\large Prepared for}\par
            \Huge \CapstoneSponsorCompany\par
            \vspace{5pt}
            {\Large\NameSigPair{\CapstoneSponsorPerson}\par}
            {\large Prepared by }\par
            Group\CapstoneTeamNumber\par
            % 5. comment out the line below this one if you do not wish to name your team
            \CapstoneTeamName\par 
            \vspace{5pt}
            {\Large
                \NameSigPair{\GroupMemberOne}\par
                \NameSigPair{\GroupMemberTwo}\par
                \NameSigPair{\GroupMemberThree}\par
            }
            \vspace{20pt}
        }
        \begin{abstract}
        % 6. Fill in your abstract    
    Our project requires the development of software to run a car infotainment system alongside a black box data log. Our software will run on an Nvidia Jetson TX2 and should be able to execute basic media and navigation tasks while also logging data from a variety of sensors. The goal is to allow people to modernize older cars with an array of fresh sensors and an interactive software interface. To accomplish this goal, we will pull vehicle data from a car?s OBDII (On Board Diagnostics II) port as well as our own sensors wired into the TX2, and feed that data to software that aggregate that data into a log. 
        \end{abstract}     
    \end{singlespace}
\end{titlepage}
\newpage
\pagenumbering{arabic}
\tableofcontents
% 7. uncomment this (if applicable). Consider adding a page break.
%\listoffigures
%\listoftables
\clearpage

% 8. now you write!
\section{Project Definition}
Currently, many companies sell aftermarket infotainment systems: third-party touchscreen devices designed to be installed in the place of a car's radio by a car owner after they have purchased their vehicle in order to provide easy access to media and navigation while driving. The majority of these systems are functionally limited. Existing aftermarket systems generally don't have access to many vehicle sensors, don't have any data logging functionality, and only provide basic infotainment functionality. Nonetheless, the market for aftermarket infotainment systems in the U.S. is relatively large, due to the abundance of older vehicles with obsolete media interfaces.\par

This presents us with our project and its goal: build an infotainment system that doubles as a black box. Our system should provide the basic media and navigation functionality of an infotainment system while also being able to log vehicle data.\par

The project specifies that the software developed to achieve this will be run on an NVIDIA Jetson TX2 embedded computer accompanied by a touchscreen used to interact with it. All of this shall be packaged in a case, creating a contained unit that can be installed into a dashboard. The project aims to achieve black box functionality by reading data off from the car?s computer and other sensors connected directly to the board and logging it in non-volatile storage. This system should also allow drivers to control external hardware wired into the infotainment system, which may include supplemental turn signals, backup cameras, dashcams, multiple GPS receivers, and an AHRS (Attitude, Heading, and Reference System) and the functionality that these devices entail. The project should also make any software developed open source, allowing for free modification and reproduction by second and third parties.\par


\section{Proposed Solution}
A software system will be developed for this project, designed to run a functional infotainment system using a simple and minimalistic user interface. The user interface should be intuitive and appeal to a broad market. Speed of the system should be emphasized in order to minimize driver distraction by offering quick access to media, navigation, vehicle data, and settings. The infotainment system will also function as a black box by reading data off of an OBD-II (On-Board Diagnostics II port), formatting it, and then saving it to a hard drive that is attached to the NVIDIA TX2. Open source third party solutions should be used to implement elements of the infotainment system where possible, simplifying the development process by eliminating the need to completely engineer and test a complex subsystem from scratch. An example of this is utilizing OpenStreetMaps for our Navigation subsystem, which provides mapping and navigation functionality. As we are using open source third party software to implement some components of our solution, our system will be an open source project.\par

\subsection{Capabilities}
The infotainment will implement FM radio support with full audio playback, tuning, and RDS (radio data system) display of channel information. Support for audio through an Auxiliary port and Bluetooth will also be included to play music from external devices. Navigation will be offered, with some support for offline capability should the user lose their mobile data signal.\par

The infotainment system will also present some of the black box?s logged information to the driver. AHRS will be used to add extra dimensions to the data logged in the black box in terms of roll, pitch, yaw, and directional heading, while also displaying this information to the driver through the touchscreen. Other devices may also be supported: dashcams may also be wired into the system in order to video the road to the black box during driving; backup cameras may also be wired to display video to the driver when reversing. All devices and sensors included in the solution will either be wired directly into the NVIDIA TX2 or read from a vehicle's OBDII port. As we are developing a system that has visual display capabilities and is designed to be installed in a vehicle, we have an obligation to limit driver distraction.\par

\section{Performance Metrics}
We will evaluate our performance on this project by examining whether or not we have fulfilled a variety of requirements that will be divided into "necessities" and "superlatives".\par
"Necessities" outline the minimum functionality that is expected of our infotainment system and black box requested by our client; not fulfilling these will amount to failure of the project; completion of all necessities amounts to the completion of the project. "Necessities" will be considered our base requirements in software development. An example of a "necessity" requirement would be: implementation of FM (frequency modulation) radio with functional tuning, audio playback, and display of RDS (radio data system) data onto the touchscreen for the current radio channel.\par
"Superlatives" will outline additional features that are not explicitly required to fulfill the requested functionality of the project, but are desirable (or nice-to-have) and will make our system more capable and versatile.\par

\subsection{Fulfillment}
Successful implementation of necessities and superlatives can be evaluated based on the presence or absence of the feature with regard to the requirements specified. This is because there is little to no middle ground with much of the functionality; for example, data logging cannot "somewhat work": it either functions or it doesn't. This makes it easier to write requirements, test the implementation of said requirements, and then subsequently gauge the completeness of the implementation.

\end{document}