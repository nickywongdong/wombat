\documentclass[onecolumn, draftclsnofoot,10pt, compsoc]{IEEEtran}
\usepackage{graphicx}
\usepackage{url}
\usepackage{setspace}

\usepackage{geometry}
\geometry{textheight=9.5in, textwidth=7in}

% 1. Fill in these details
\def \CapstoneTeamName{		Team Wombat}
\def \CapstoneTeamNumber{		15}
\def \GroupMemberOne{			Victor Li}
\def \GroupMemberTwo{			Ryan Crane}
\def \GroupMemberThree{			Nicholas Wong}
\def \CapstoneProjectName{		NVIDIA Jetson TX2 Infotainment and Black Box}
\def \CapstoneSponsorCompany{	}
\def \CapstoneSponsorPerson{		Kevin McGrath}

% 2. Uncomment the appropriate line below so that the document type works
\def \DocType{		Problem Statement
				%Requirements Document
				%Technology Review
				%Design Document
				%Progress Report
				}
			
\newcommand{\NameSigPair}[1]{\par
\makebox[2.75in][r]{#1} \hfil 	\makebox[3.25in]{\makebox[2.25in]{\hrulefill} \hfill		\makebox[.75in]{\hrulefill}}
\par\vspace{-12pt} \textit{\tiny\noindent
\makebox[2.75in]{} \hfil		\makebox[3.25in]{\makebox[2.25in][r]{Signature} \hfill	\makebox[.75in][r]{Date}}}}
% 3. If the document is not to be signed, uncomment the RENEWcommand below
%\renewcommand{\NameSigPair}[1]{#1}

%%%%%%%%%%%%%%%%%%%%%%%%%%%%%%%%%%%%%%%
\begin{document}
\begin{titlepage}
    \pagenumbering{gobble}
    \begin{singlespace}
    	\includegraphics[height=4cm]{coe_v_spot1}
        \hfill 
        % 4. If you have a logo, use this includegraphics command to put it on the coversheet.
        %\includegraphics[height=4cm]{CompanyLogo}   
        \par\vspace{.2in}
        \centering
        \scshape{
            \huge CS Capstone \DocType \par
            {\large\today}\par
            \vspace{.5in}
            \textbf{\Huge\CapstoneProjectName}\par
            \vfill
            {\large Prepared for}\par
            \Huge \CapstoneSponsorCompany\par
            \vspace{5pt}
            {\Large\NameSigPair{\CapstoneSponsorPerson}\par}
            {\large Prepared by }\par
            Group\CapstoneTeamNumber\par
            % 5. comment out the line below this one if you do not wish to name your team
            \CapstoneTeamName\par 
            \vspace{5pt}
            {\Large
                \NameSigPair{\GroupMemberOne}\par
                \NameSigPair{\GroupMemberTwo}\par
                \NameSigPair{\GroupMemberThree}\par
            }
            \vspace{20pt}
        }
        \begin{abstract}
        % 6. Fill in your abstract    
    The abundance of vehicle sensors, necessity of data recording, and prevalence of infotainment systems in modern automobiles creates a new avenue of third-party aftermarket hardware that can combine the functionality of a black box and that of an infotainment system, utilizing a thorough set of modern sensors.\par
	To date, no existing aftermarket systems offer such infotainment and black box capabilities in a complete package. This untapped avenue of hardware can be capitalized upon by the creation of a software system on a hardware component(s) that merges the data logging capabilities of a black box system with the media, navigation, and vehicle setting management capabilities of an infotainment system, complete in a unit that can be installed in a vehicle and allows the driver to access relevant recorded data.\par
	It will make drivers more knowledgeable about their vehicle by giving them easy access to an understandable range of vehicle information that may be useful in a range of situations on and off the road, whilst also offering drivers with older vehicles, coupled with low-volume car manufacturers, a complete piece of hardware that can modernize their vehicles' infotainment, connectivity, and data logging capabilities.\par
        \end{abstract}     
    \end{singlespace}
\end{titlepage}
\newpage
\pagenumbering{arabic}
\tableofcontents
% 7. uncomment this (if applicable). Consider adding a page break.
%\listoffigures
%\listoftables
\clearpage

% 8. now you write!
\section{Project Definition}
Currently, many companies sell aftermarket infotainment systems: third-party touchscreen devices designed to be installed in the place of a car's radio by a carowner after they have purchased their vehicle in order to provide easy access to media and navigation while driving. However, the majority of these systems are standalone and functionally limited. Existing aftermarket systems generally don't have access to many vehicle sensors if at all, don't have any data logging functionality, and do little else than provide basic infotainment functionality. Nonetheless, the market for aftermarket infotainment systems in the U.S. is relatively large, as consumers here drive some of the oldest vehicles in the world with respect to average vehicle age. Many cars are not equipped with infotainment at all, limiting the functionality of millions of vehicles; an issue that many drivers are eager to rectify.\par
This presents us with our project and its goal: integrating an infotainment system with an automotive black box (a device that records snippets of a vehicle's data that originates from a modern vehicle's numerous sensors), such that our system provides the base functionality of both components (audio playback, navigation, data logging, etc.) whilst also offering some vehicle control, interactive, and informational capabilities that exceeds that of a standard aftermarket infotainment system.\par
The project necessitates that the software developed to achieve this be installed to NVIDIA Jetson TX2 embedded computer\--NVIDIA's latest automotive computer system\--accompanied by a touchscreen used to interact with it. All of this shall be packaged in a case, creating a contained unit that can be installed into a dashboard. The project aims to achieve black box functionality by reading data off of some vehicle port and logging that to some sort of format to a data storage medium. This system should also allow drivers to control external hardware wired into the infotainment system, which may include supplemental turn signals, backup cameras, dashcams, multiple GPS receivers, and an AHRS (Attitude, Heading, and Reference System) and the functionality that these devices entail. The project should also make any software developed open source, allowing for free modification and reproduction by second and third parties.\par
Overall, the infotainment and black box developed using the NVIDIA TX2 hardware should be more extensible and powerful than existing infotainment systems, giving drivers access to more features that may include extra control functionality or interactivity with their vehicle and its data in a way that most vehicular infotainment systems don't offer. Doing this will give us a cohesive aftermarket infotainment and black box solution that has never been encountered before.\par

\section{Proposed Solution}
A software system will developed for this project, designed to run a functional infotainment system using a simple and minimalistic user interface (the visual layout of the touchscreen and how the user interacts with it to access the system). The user interface should be easy to use and be oriented towards the mass-market in order to make it appeal to as many potential customers as possible. Speed of the system should be emphasized in order to minimize driver distraction by offering quick access to media, navigation, vehicle data, and settings. The infotainment system will also function as a black box by reading data off of an OBD-II On-Board Diagnostics II port), formatting it, and then saving it to a hard drive that is attached to the NVIDIA TX2. Open source (freely modifiable and accessible) third party solutions should be used to implement elements of the infotainment system where possible, simplifying the development process by eliminating the need to completely engineer and test a complex subsystem from scratch, especially as there may be plentiful existing solutions. An example of this is utilizing OpenStreetMaps for our Navigation subsystem, which provides mapping and navigation functionality.\par

\subsection{Capabilities}
The infotainment will implement FM radio support with full audio playback, tuning, and RDS (radio data system) display of channel information. Mobile phone Auxiliary input through a 3.5mm headphone jack will also be implemented, allowing playback through that medium as well. Additionally, Bluetooth Streaming, allowing for audio playback and control via a Bluetooth-connected device, is also a possibility. Navigation will be offered, and support offline capability should the user lose their mobile data signal. The infotainment system will also expand on the normal capability of sensors by both logging extra data in the black box as well as presenting some of the information from those sensors to the driver. GPS used for navigation will likely support multiple GPS receivers in order to triangulate the location of larger vehicles. AHRS will be used to add extra dimensions to the data logged in the black box in terms of roll, pitch, yaw, and directional heading, whilst also displaying this information to the driver in near real-time through the touchscreen. Other devices may also be supported: dashcams may also be wired into the system in order to record the road during driving, logging video to the black box to make the data recorded more comprehensive; backup cameras may also be wired to display video to the driver when reversing, but need not necessarily record data to the black box. All devices and sensors included in the final design will either be wired directly into the NVIDIA TX2 or read from a vehicle's OBD-II port (support for a vehicle's Controller Area Network bus\--or CAN bus\--is also possible).\par

\subsection{Legal, Distribution, and Copyright}
As we are developing a system that has data logging capabilities and is designed to be installed into a vehicle, we have legal obligations. Federal privacy laws and automotive regulations should be followed with regards to anything implemented in this solution, as ensuring the privacy and safety of potential users should be paramount.\par
As the solution developed ideally creates open source code, the software we develop will utilize an open source license. Our development team should retain the overall intellectual property in the implementation, however not necessarily all the code. This will be accomplished using an Apache 2.0 license or alike.\par

\section{Performance Metrics}
We will evaluate our performance on this project by examining whether or not we have fulfilled a variety of requirements that will be divided into "necessities" and "superlatives".\par
"Necessities" outline the minimum functionality that is expected of our infotainment system and black box requested by our client; not fulfilling these will amount to failure of the project; completion of all necessities amounts to the completion of the project. "Necessities" will be considered our base requirements in software development. An example of a necessity requirement would be: implementation of FM (frequency modulation) radio such that there is tuning capability, correct audio playback, display of RDS (radio data system) data onto the touchscreen for the current radio channel.\par
"Superlatives" will outline additional features that are not explicitly required to fulfill the requested functionality of the project, but are desirable (or nice-to-have) and will make our system more capable and versatile.\par

\subsection{Fulfillment}
Both metrics of necessities and superlatives will be graded on a yes/no basis with regards to functionality, as that is the best way to assess elements of the system. This is because there is little to no middle ground with much of the functionality; for example, navigation cannot "somewhat work": it either functions or it doesn't function. This makes it easier to write requirements, test the implementation of said requirements, and then subsequently gauge the completeness of the  implementation used to fulfill the requirements, enabling our development process to be much more streamlined and holistic.\par

\end{document}